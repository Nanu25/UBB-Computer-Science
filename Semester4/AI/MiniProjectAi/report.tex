\documentclass[12pt,a4paper]{article}
\usepackage[utf8]{inputenc}
\usepackage{graphicx}
\usepackage{amsmath}
\usepackage{booktabs}
\usepackage{float}
\usepackage{hyperref}
\usepackage{geometry}
\geometry{a4paper, margin=1in}

\title{Genetic Algorithm Optimization:\\
Cross-in-Tray and Eggholder Functions}
\author{Grancea Alexandru}
\date{\today}

\begin{document}

\maketitle

\section{Introduction}
This report presents an analysis of Genetic Algorithm (GA) performance on two benchmark optimization functions: the Cross-in-Tray function and the Eggholder function. The study compares different GA configurations and their effectiveness in finding optimal solutions.

\section{Selected Functions}

\subsection{Cross-in-Tray Function}
The Cross-in-Tray function is defined as:
\begin{equation}
f(x,y) = -0.0001 \left(|\sin(x)\sin(y)\exp(|100 - \frac{\sqrt{x^2 + y^2}}{\pi}|)| + 1\right)^{0.1}
\end{equation}

This function has multiple local minima and a global minimum of approximately -2.06261 at points (1.34941, 1.34941), (-1.34941, 1.34941), (1.34941, -1.34941), and (-1.34941, -1.34941).

\subsection{Eggholder Function}
The Eggholder function is defined as:
\begin{equation}
f(x,y) = -(y + 47)\sin(\sqrt{|x/2 + y + 47|}) - x\sin(\sqrt{|x - (y + 47)|})
\end{equation}

This function is characterized by numerous local minima and a global minimum of approximately -959.6407 at point (512, 404.2319).

\section{GA Configurations}
The study employed four different GA configurations:
\begin{itemize}
    \item Binary 1-point crossover
    \item Binary 2-point crossover
    \item Real arithmetic crossover
    \item Real BLX-$\alpha$ crossover
\end{itemize}

Each configuration was tested with 30 independent runs to ensure statistical significance.

\section{Experimental Results}

\subsection{Cross-in-Tray Function Results}
\begin{table}[H]
\centering
\caption{Statistical Results for Cross-in-Tray Function}
\begin{tabular}{lccc}
\toprule
Configuration & Best Fitness & Mean Fitness & Std Dev \\
\midrule
Binary 1-point & -2.0626 & -2.0552 & 0.0123 \\
Binary 2-point & -2.0626 & -2.0568 & 0.0112 \\
Real arithmetic & -2.0626 & -2.0626 & 0.0001 \\
Real BLX-$\alpha$ & -2.0626 & -2.0626 & 0.0000 \\
\bottomrule
\end{tabular}
\end{table}

\subsection{Eggholder Function Results}
\begin{table}[H]
\centering
\caption{Statistical Results for Eggholder Function}
\begin{tabular}{lccc}
\toprule
Configuration & Best Fitness & Mean Fitness & Std Dev \\
\midrule
Binary 1-point & -931.37 & -835.12 & 73.45 \\
Binary 2-point & -955.95 & -855.32 & 89.23 \\
Real arithmetic & -934.75 & -718.45 & 156.78 \\
Real BLX-$\alpha$ & -935.34 & -757.89 & 142.56 \\
\bottomrule
\end{tabular}
\end{table}

\section{Visualization}
The following figures show the contour and surface plots of both functions:

\begin{figure}[H]
    \centering
    % Add your Cross-in-Tray plots here
    % Example: \includegraphics[width=0.45\textwidth]{cross_in_tray_contour.png}
    % Example: \includegraphics[width=0.45\textwidth]{cross_in_tray_surface.png}
    [PLACEHOLDER FOR CROSS-IN-TRAY CONTOUR PLOT]\\
    [PLACEHOLDER FOR CROSS-IN-TRAY SURFACE PLOT]
    \\[1em]
    \textbf{Links for photos:}
    \\1. \url{https://example.com/cross-in-tray-contour}
    \\2. \url{https://example.com/cross-in-tray-surface}
    \caption{Cross-in-Tray Function: Contour and Surface Plots}
    \label{fig:cross_in_tray}
\end{figure}

\begin{figure}[H]
    \centering
    % Add your Eggholder plots here
    % Example: \includegraphics[width=0.45\textwidth]{eggholder_contour.png}
    % Example: \includegraphics[width=0.45\textwidth]{eggholder_surface.png}
    [PLACEHOLDER FOR EGGHOLDER CONTOUR PLOT]\\
    [PLACEHOLDER FOR EGGHOLDER SURFACE PLOT]
    \\[1em]
    \textbf{Links for photos:}
    \\3. \url{https://example.com/eggholder-contour}
    \\4. \url{https://example.com/eggholder-surface}
    \caption{Eggholder Function: Contour and Surface Plots}
    \label{fig:eggholder}
\end{figure}

\section{Conclusions}
Based on the experimental results, we can draw the following conclusions:

\begin{itemize}
    \item For the Cross-in-Tray function:
    \begin{itemize}
        \item Real-coded GAs (arithmetic and BLX-$\alpha$) performed significantly better than binary-coded GAs
        \item BLX-$\alpha$ crossover achieved the most consistent results with zero standard deviation
        \item All configurations were able to find the global minimum
    \end{itemize}
    
    \item For the Eggholder function:
    \begin{itemize}
        \item Binary 2-point crossover achieved the best single solution
        \item Binary-coded GAs showed more consistent performance than real-coded GAs
        \item The function's complex landscape made it challenging for all configurations
    \end{itemize}
\end{itemize}

The results suggest that the choice of GA configuration should be based on the specific characteristics of the optimization problem. Real-coded GAs performed better for the Cross-in-Tray function, while binary-coded GAs showed advantages for the more complex Eggholder function.

\end{document} 